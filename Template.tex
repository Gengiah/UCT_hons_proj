\documentclass[12pt,preprint, authoryear]{elsarticle}

\usepackage{lmodern}
%%%% My spacing
\usepackage{setspace}
\setstretch{1.5}
\DeclareMathSizes{12}{14}{10}{10}

% Wrap around which gives all figures included the [H] command, or places it "here". This can be tedious to code in Rmarkdown.
\usepackage{float}
\let\origfigure\figure
\let\endorigfigure\endfigure
\renewenvironment{figure}[1][2] {
    \expandafter\origfigure\expandafter[H]
} {
    \endorigfigure
}

\let\origtable\table
\let\endorigtable\endtable
\renewenvironment{table}[1][2] {
    \expandafter\origtable\expandafter[H]
} {
    \endorigtable
}


\usepackage{ifxetex,ifluatex}
\usepackage{fixltx2e} % provides \textsubscript
\ifnum 0\ifxetex 1\fi\ifluatex 1\fi=0 % if pdftex
  \usepackage[T1]{fontenc}
  \usepackage[utf8]{inputenc}
\else % if luatex or xelatex
  \ifxetex
    \usepackage{mathspec}
    \usepackage{xltxtra,xunicode}
  \else
    \usepackage{fontspec}
  \fi
  \defaultfontfeatures{Mapping=tex-text,Scale=MatchLowercase}
  \newcommand{\euro}{€}
\fi

\usepackage{amssymb, amsmath, amsthm, amsfonts}

\usepackage[round]{natbib}
\bibliographystyle{natbib}
\def\bibsection{\section*{References}} %%% Make "References" appear before bibliography
\usepackage{longtable}
\usepackage[margin=2cm,bottom=4cm,top=2.5cm, includefoot]{geometry}
\usepackage{fancyhdr}
\usepackage[bottom, hang, flushmargin]{footmisc}
\usepackage{graphicx}
\numberwithin{equation}{section}
\numberwithin{figure}{section}
\numberwithin{table}{section}
\setlength{\parindent}{0cm}
\setlength{\parskip}{1.3ex plus 0.5ex minus 0.3ex}
\usepackage{textcomp}
\renewcommand{\headrulewidth}{0.2pt}
\renewcommand{\footrulewidth}{0.3pt}

\usepackage{array}
\newcolumntype{x}[1]{>{\centering\arraybackslash\hspace{0pt}}p{#1}}

%%%%  Remove the "preprint submitted to" part. Don't worry about this either, it just looks better without it:
\journal{Journal of Finance}

 \def\tightlist{} % This allows for subbullets!

\usepackage{hyperref}
\hypersetup{breaklinks=true,
            bookmarks=true,
            colorlinks=true,
            citecolor=blue,
            urlcolor=blue,
            linkcolor=blue,
            pdfborder={0 0 0}}

\urlstyle{same}  % don't use monospace font for urls
\setlength{\parindent}{0pt}
\setlength{\parskip}{6pt plus 2pt minus 1pt}
\setlength{\emergencystretch}{3em}  % prevent overfull lines
\setcounter{secnumdepth}{5}

%%% Use protect on footnotes to avoid problems with footnotes in titles
\let\rmarkdownfootnote\footnote%
\def\footnote{\protect\rmarkdownfootnote}
\IfFileExists{upquote.sty}{\usepackage{upquote}}{}

%%% Include extra packages specified by user
% Insert custom packages here as follows
% \usepackage{tikz}

\begin{document}

\begin{frontmatter}  %

\title{Profitability of moving average trends in the Top40}

\author[Add1]{Nico Katzke}
\ead{nfkatzke@gmail.com}

\author[Add1,Add2]{John Smith}
\ead{John Smith@gmail.com}

\author[Add1,Add2]{John Doe}
\ead{JohnSmith@gmail.com}



\address[Add1]{Bureau for Economic Research, Stellenbosch University, South Africa}
\address[Add2]{Some other Institution, Cape Town, South Africa}

\cortext[cor]{Corresponding author: Nico Katzke}

\begin{abstract}
\small{
Abstract to be written here. The abstract should not be too long and
should provide the reader with a good understanding what you are writing
about. Academic papers are not like novels where you keep the reader in
suspense. To be effective in getting others to read your paper, be as
open and concise about your findings here as possible. Ideally, upon
reading your abstract, the reader should feel he / she must read your
paper in entirety.
}
\end{abstract}

\vspace{1cm}

\begin{keyword}
\footnotesize{
Multivariate GARCH \sep Kalman Filter \sep Copula \\ \vspace{0.3cm}
\textit{JEL classification} L250 \sep L100
}
\end{keyword}
\vspace{0.5cm}
\end{frontmatter}



%________________________
% Header and Footers
%%%%%%%%%%%%%%%%%%%%%%%%%%%%%%%%%
\pagestyle{fancy}
\chead{}
\rhead{}
\lfoot{}
\rfoot{\footnotesize Page \thepage\\}
\lhead{}
%\rfoot{\footnotesize Page \thepage\ } % "e.g. Page 2"
\cfoot{}

%\setlength\headheight{30pt}
%%%%%%%%%%%%%%%%%%%%%%%%%%%%%%%%%
%________________________

\headsep 35pt % So that header does not go over title




\section{Introduction}\label{introduction}

The literature review aims to provide a better understanding of moving
average trend lines in the context of the South African market in
particular the JSE Top40 index. The aim of the literature review is the
analysis of simple and exponential moving averages and how they are used
in generating profits. The literature will explore the profitability of
moving average trend lines in other countries and the feasibility of
moving average trend lines in the South African environment. ``The Top40
is South Africa's best-known index. It includes the 40 largest companies
listed on the JSE. This is the index that most people monitor as an
overall benchmark for the local exchange.'' (JSE, 2017) These 40
companies are the largest in terms of market capitalization which
provides a representation of the South African market as a whole.
Performing moving average trend line analysis on the Top40 index will
generate results representing the profitability of Moving average trend
lines in the South African market. Moving Average trend lines are a form
of technical analysis. Technical analysis involves using the volume and
latest stock prices to determine models and technical trading indicators
for a given set of data. Moving averages are the most widely known and
used by practitioners and financial traders in the markets. (Sobreiro,
et al., 2016) Charles H. Dow believed stock market prices could provide
information on the overall market. This lead to Dow theory which was
introduced in the late 1800's starting what is known as technical
analysis. Technical analysis is used in the financial industry amongst
different market participants. The introduction of the Efficient Market
Hypothesis proposed by Fama (1970) lead to a decline in the use of
technical analysis and the disbelief that profits could be generated
from technical analysis. (Sobreiro, et al., 2016)It was only in 1992
when a paper done by William Brock, Josef Lakonishok and Blake LeBaron
provided significant evidence in justifying the use of technical
analysis. (Fong, W.M \& Yong, L.H.M, 2005). The paper lead to the
resurgence in testing technical analysis in different markets and
determining whether it is actually profitable. There is currently no
research on moving average trend lines in the Top40. A paper was done on
the 5 emerging national economies of Brazil, Russia, India, China and
South Africa, (BRICS) The paper analyzed South Africa but did not cover
the South African market extensively. The South African stock market is
described in a paper done in 2003 as not weak form market efficient
suggesting profitable opportunities Appiah-Kusi and Menyah
(\protect\hyperlink{ref-appiah2003return}{2003}). The lack of
information involving moving average trend lines in the South African
market and the potential for profits emphasizes the importance of the
research question. Moving averages provide a simple method to
determining when stocks should be bought or sold. The main strategy of
moving averages is to determine a trend in share prices. This trend is
then plotted in the form of a moving average trend line. The short/sell
signal occurs when the price is below the moving average and long/buy
signal occurs when the price is above the moving average. A moving
average trend line is profitable if it provides excess returns when
compared to the buy-and-hold strategy in the JSE Top40 index. The
buy-and-hold strategy is just the returns you get from investing in a
specific index without active management of buying and selling in the
JSE market. (Move to methodology)

\section{The Efficient Market
Hypothesis}\label{the-efficient-market-hypothesis}

The paper rewritten by (Fama,1991) refers to the Efficient Market
Hypothesis as the asset prices fully reflecting all available
information. The Efficient Market Hypothesis can be split into 3 forms
based on the definition of the information: Weak Form EMH: Prices
reflect all information in the past price history Semi-Strong Form EMH:
Prices reflect all publicly available information Strong Form EMH:
Prices reflect all information, public and private. (Park \& Irwin,
2004) If all information is available in the prices, then additional
information such as moving averages shouldn't provide any financial
profitability. The Weak form efficient market hypothesis states that
there is no justification for technical analysis. This implies that
markets are efficient and arbitrage should not be attainable through
technical analysis. (Fama, 1991). The weak form of the efficient market
hypothesis has been the paradigm in describing the behaviour of prices
in speculative market. A paper done in 1988 proved the non-existence of
random walks in the stock market using the NYSE-AMEX index. (Lo \&
MacKinlay, 1988) The Efficient Market Hypothesis is one of the main
reasons why individuals do not regard technical analysis as useful and
has lead market participants to avoid technical analysis. Technical
analysis is still used by market participants. Moving averages might not
provide exact profits but they do provide other information about the
data. ``Moving Average are a type of smoothing method for reducing, or
cancelling random variation inherent in data taken over time. When
applied properly, this technique reveals more clearly the underlying
trend, seasonal and cyclic approach components in the data.'' (Okkels,
2014) This definition implies that moving average can still be used in
understanding the direction of the market even if it cannot be used
directly in beating the buy-and-hold strategy.

\section{Methodology}\label{methodology}

The 2 main Moving Averages discussed are simple moving average and
exponential moving average. Simple moving averages is the sum of latest
stock prices divided by the number of stock prices: Simple moving
average calculation

{[}MATHHHH{]}

Where: \(P_t\) is the closing price of the stock in t period. \(n\) is
the relative position of the current period observed; and k is the
number of periods included un the SMA calculation;

Simple averages are the easiest to interpret while exponential moving
averages provide stronger predictive ability in market prices.
Exponential moving averages focus on the most recent values and thus are
similar to weighted moving averages.

Exponential moving average calculation where: {[}MATH{]} is the closing
price of the stock in the previous period. {[}MATH{]} is the EMA in the
previous period. n is the relative position of the current period
observed; and k is the number of periods included in the EMA
calculation;

There are 2 other moving averages which are not as common as the simple
and exponential moving average. These are weighted and adaptive moving
average. Weighted moving averages are similar to exponential moving
averages but place specific importance on different stock prices.
Kaufman adaptive moving average is a moving average that accounts for
volatility. (Sobreiro, et al., 2016)The literature review will focus
mainly on simple and exponential moving average trend lines.

The simple and exponential moving averages are used in different ways to
predict the buy and sell signals in the price history. Exponential
moving average gives more weight to recent stock prices while simple
provides equal weighting for all stock prices in a given time period.
(Okkels, 2014). One method is the comparison of short and long run
simple moving averages. Short and long run moving averages refers to the
specific time frame in which the moving average is calculated. In a
paper done on South Asian markets, short is referred to as 10 or 20 day
moving averages, while long run is referred to as 50 or 200 day moving
averages. (Ming-Ming \& Siok-Hwa, 2006)

Moving Average Convergence Divergence is the formal name for comparing a
short and long run exponential moving average. The most common time
periods for the short moving average is 9 days while a long moving
average has a time period of 26 days. (Okkels, 2014)

Moving average techniques require buy and sell signals after comparing
short and long run simple or exponential moving averages. (Gunasekerage
\& Power, 2001)). The paper discusses 2 moving average techniques which
are: Variable length moving average and fixed length moving average. The
variable length moving average rule generates a buy-and-sell signal
every day. Variable length moving average signals to buy when short
moving average is above the long moving average by an amount larger than
the band. (Brock, et al., 1992)The band is some percentage threshold
such that the difference in short and long run Moving average is still
significant. Brock uses a threshold of 1\%. The variable length moving
average will change once the short moving average exceeds the long
moving average by the 1\% threshold. The crossing over of Moving
averages is referred to as the dual moving average crossover. (Ming-Ming
\& Siok-Hwa, 2006). The fixed length moving average holds a moving
average for a fixed period of time. This is a 10-day holding period in
Brock (1992) and Ming-Ming, L \& Siok-Hwa, L (2005). During this 10-day
period signals do not change the fixed length moving average till the
end of the holding period, where cumulative profits are calculated.
Fixed length moving average focuses on crossing over of long and short
moving averages. A sell (buy) signal occurs when the short (long) moving
average crosses the long (short) moving average from above.

The daily closing price index is used as the short term moving average
in another paper. (Ming-Ming \& Siok-Hwa, 2006) The long term moving
average varies between the time period of 20, 60,120,180,240 days. The
short-term moving average is compared to the long term moving average in
calculating profits and the band threshold is 1\%. Multiple long moving
averages are compared to the 1-day short moving average. The best
strategy was using a 60-day long moving average. The results indicated
variable moving averages were more profitable than fixed moving average
and both provided significant returns.

A Study done on the South Asian stock market used 9 different moving
average rules. (1,50,0),(1,100,0),
(1,150,0),(1,200,0),(2,100,0),(2,150,0),(5,200,0) and (1,50,0). The 1st
number in the bracket represents the short length moving average, the
2nd number in the bracket represents the long moving average and the 3rd
number indicated the band threshold percentage. (Gunasekerage \& Power,
2001)The study was compared to the study done by Brock et al (1992). The
study analysed the log returns of prices in the index as it is easier to
interpret and calculate mean returns.

The Methodology that should be incorporated in answering the research
question should be the application of dual moving average crossover and
moving average convergence divergence. This allows for short and long
term moving averages while comparing simple and exponential moving
averages. The literature reviewed suggests using a short moving average
of either 1 or 10 days and long moving average of either 50,100 or 200
days. The band threshold used in all studies was either 1\% or 0. The
time frame that could be used is 2001-2016.

\section{Empirical results}\label{empirical-results}

The research performed on the South African market showed positive
results for exponential moving averages and negative results for simple
moving average. This suggest exponential moving average might be more
useful in analysing the South African Market. (Sobreiro, et al., 2016)

Given that there were no other studies done specifically on the South
African market we can look at results from all over the world focusing
mainly on emerging markets which are similar to the South African
context. (Sobreiro, et al., 2016)

Brock (1992) analysed fixed and variable moving averages in the Dow
Jones industrial Average Index using 90 years of data. (1897-1986). Both
the fixed and variable moving averages provided excess returns compared
to the buy-and-hold strategy. These returns did not include transaction
costs. (Bessembinder \& Chan, 1998)A study done on 6 equity indices in
Asia provided results indicating strong predictive power of moving
averages in Malaysian, Thailand and Taiwan stock markets. Another study
done on the Financial Times industrial Ordinary Index did not provide
excess returns after a transaction cost of 1\% was considered. The
problem with Brocks excess returns from trading rules is that
transaction costs are not included, thus are not sufficient in proving
that technical trading is profitable. (Park and Irwin, 2007)

The paper written on South Asian Markets also provided results
confirming the profitability of moving average trend lines.
(Bessembinder \& Chan, 1998)The study obtained stock price index data
from (1975-1989). The returns are computed as changes in log price
indices. The VLMA provided more buy and sell signals while the FLMA
generated more profit. The transaction costs were reported as break even
costs in order to calculate the highest percentage of profits available
for transaction costs. The average transaction cost is 1.57\% for the
whole sample. If the transaction cost was 1\% then moving averages would
have been profitable.

The empirical results from past literature provides evidence in moving
average trend lines being profitable. The study will prove or disprove
if this is the case in the JSE Top40.

\section{Conclusion}\label{conclusion}

The literature reviewed is inconclusive on the exact impact of Moving
average trend lines and if it can beat the buy-and-hold strategy. The
empirical results provide evidence in both proving and disproving the
profitability of moving average trend lines after considering
transaction costs. There are a few methods to determining the predictive
power of moving averages trend lines. The most common is comparing short
and long run fixed and variable moving averages. The most common
transaction cost viewed in most papers was 1\% and a band threshold of
1\%. The literature reviewed has provided a guideline in testing if
moving average trend lines are profitable in the JSE Top40.

\hypertarget{refs}{}
\hypertarget{ref-appiah2003return}{}
Appiah-Kusi, Joe, and Kojo Menyah. 2003. ``Return Predictability in
African Stock Markets.'' \emph{Review of Financial Economics} 12 (3).
Elsevier: 247--70.

% Force include bibliography in my chosen format:
\newpage
\nocite{*}
\bibliography{}





\end{document}
